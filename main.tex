\title{Untitled document}
\author{
    Alex Tsui
}
\date{\today}

%\documentclass[12pt]{report}
\documentclass[12pt]{article}
%\documentclass[tikz,border=10pt]{standalone}
%\usepackage{tikz}
%\usetikzlibrary{quotes,angles}
\usepackage{amsmath}
\usepackage{amsthm}
\usepackage{amssymb}
\usepackage{hyperref}
\usepackage{graphicx}
\usepackage{underscore}
\usepackage{makeidx}
\usepackage{ulem}
\usepackage{xcolor,listings}
\usepackage{csquotes}
\usepackage{hyperref} % this should be last package loaded
\newcommand{\sgn}{\operatorname{sgn}}
\lstloadlanguages{[LaTeX]TeX, [primitive]TeX,fortran}
\lstset{language={[LaTeX]TeX},
      escapeinside={{(*@}{@*)}},
       gobble=0,
       stepnumber=1,numbersep=5pt,
       numberstyle={\footnotesize\color{gray}},%firstnumber=last,
      breaklines=true,
      framesep=5pt,
      basicstyle=\small\ttfamily,
      showstringspaces=false,
      keywordstyle=\ttfamily\textcolor{blue},
      stringstyle=\color{orange},
      commentstyle=\color{black},
      rulecolor=\color{gray!10},
      breakatwhitespace=false,
      showspaces=false,  % shows spacing symbol
      backgroundcolor=\color{gray!15}}
\SetBlockEnvironment{quotation}
\makeindex

\begin{document}
\maketitle
\section{Introduction}
Today, I installed TikZ. Check out the results in Figures
\ref{fig:angle} and \ref{fig:torus-triangulation-minimal}.

\begin{figure}[h!]
    \centering
    \includegraphics[width=\columnwidth]{images/angle.pdf}
    \caption{TikZ is awesome}
    \label{fig:angle}
\end{figure}

\begin{figure}[h!]
    \centering
    \includegraphics[width=\columnwidth]{images/torus-triangulation-minimal.pdf}
    \caption{TikZ is awesome}
    \label{fig:torus-triangulation-minimal}
\end{figure}

What I had to do was pretty minimal. On Ubuntu 14.04, I already had a
\texttt{texlive} installation, but TikZ 3.0 is not yet available as a package.
Then,

\begin{enumerate}
\item I went to \cite{url:pgf} to get the archive, extracted in
\texttt{\textasciitilde/texmf}, which is in the default search path for \LaTeX
styles.
\item Ran \texttt{texhash} to update the cache.
\end{enumerate}

The TikZ figures are generated from standalone tex files, kept in the
\texttt{images/} subdirectory. You should specify additional TikZ figures in
\texttt{Makefile.inc} in the root directory. Calling \texttt{make} at the root
directory will then trigger them to be built. The resulting pdfs are embedded
in the main document as usual figures.

%\section{REFERENCES}
\label{sec:ref}
\bibliography{main}
\bibliographystyle{plain}

\end{document}
This is never printed
